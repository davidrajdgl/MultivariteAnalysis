\documentclass[]{article}
\usepackage{lmodern}
\usepackage{amssymb,amsmath}
\usepackage{ifxetex,ifluatex}
\usepackage{fixltx2e} % provides \textsubscript
\ifnum 0\ifxetex 1\fi\ifluatex 1\fi=0 % if pdftex
  \usepackage[T1]{fontenc}
  \usepackage[utf8]{inputenc}
\else % if luatex or xelatex
  \ifxetex
    \usepackage{mathspec}
  \else
    \usepackage{fontspec}
  \fi
  \defaultfontfeatures{Ligatures=TeX,Scale=MatchLowercase}
\fi
% use upquote if available, for straight quotes in verbatim environments
\IfFileExists{upquote.sty}{\usepackage{upquote}}{}
% use microtype if available
\IfFileExists{microtype.sty}{%
\usepackage{microtype}
\UseMicrotypeSet[protrusion]{basicmath} % disable protrusion for tt fonts
}{}
\usepackage[margin=1in]{geometry}
\usepackage{hyperref}
\hypersetup{unicode=true,
            pdftitle={Practicals 5},
            pdfauthor={David Raj Micheal},
            pdfborder={0 0 0},
            breaklinks=true}
\urlstyle{same}  % don't use monospace font for urls
\usepackage{color}
\usepackage{fancyvrb}
\newcommand{\VerbBar}{|}
\newcommand{\VERB}{\Verb[commandchars=\\\{\}]}
\DefineVerbatimEnvironment{Highlighting}{Verbatim}{commandchars=\\\{\}}
% Add ',fontsize=\small' for more characters per line
\usepackage{framed}
\definecolor{shadecolor}{RGB}{248,248,248}
\newenvironment{Shaded}{\begin{snugshade}}{\end{snugshade}}
\newcommand{\KeywordTok}[1]{\textcolor[rgb]{0.13,0.29,0.53}{\textbf{#1}}}
\newcommand{\DataTypeTok}[1]{\textcolor[rgb]{0.13,0.29,0.53}{#1}}
\newcommand{\DecValTok}[1]{\textcolor[rgb]{0.00,0.00,0.81}{#1}}
\newcommand{\BaseNTok}[1]{\textcolor[rgb]{0.00,0.00,0.81}{#1}}
\newcommand{\FloatTok}[1]{\textcolor[rgb]{0.00,0.00,0.81}{#1}}
\newcommand{\ConstantTok}[1]{\textcolor[rgb]{0.00,0.00,0.00}{#1}}
\newcommand{\CharTok}[1]{\textcolor[rgb]{0.31,0.60,0.02}{#1}}
\newcommand{\SpecialCharTok}[1]{\textcolor[rgb]{0.00,0.00,0.00}{#1}}
\newcommand{\StringTok}[1]{\textcolor[rgb]{0.31,0.60,0.02}{#1}}
\newcommand{\VerbatimStringTok}[1]{\textcolor[rgb]{0.31,0.60,0.02}{#1}}
\newcommand{\SpecialStringTok}[1]{\textcolor[rgb]{0.31,0.60,0.02}{#1}}
\newcommand{\ImportTok}[1]{#1}
\newcommand{\CommentTok}[1]{\textcolor[rgb]{0.56,0.35,0.01}{\textit{#1}}}
\newcommand{\DocumentationTok}[1]{\textcolor[rgb]{0.56,0.35,0.01}{\textbf{\textit{#1}}}}
\newcommand{\AnnotationTok}[1]{\textcolor[rgb]{0.56,0.35,0.01}{\textbf{\textit{#1}}}}
\newcommand{\CommentVarTok}[1]{\textcolor[rgb]{0.56,0.35,0.01}{\textbf{\textit{#1}}}}
\newcommand{\OtherTok}[1]{\textcolor[rgb]{0.56,0.35,0.01}{#1}}
\newcommand{\FunctionTok}[1]{\textcolor[rgb]{0.00,0.00,0.00}{#1}}
\newcommand{\VariableTok}[1]{\textcolor[rgb]{0.00,0.00,0.00}{#1}}
\newcommand{\ControlFlowTok}[1]{\textcolor[rgb]{0.13,0.29,0.53}{\textbf{#1}}}
\newcommand{\OperatorTok}[1]{\textcolor[rgb]{0.81,0.36,0.00}{\textbf{#1}}}
\newcommand{\BuiltInTok}[1]{#1}
\newcommand{\ExtensionTok}[1]{#1}
\newcommand{\PreprocessorTok}[1]{\textcolor[rgb]{0.56,0.35,0.01}{\textit{#1}}}
\newcommand{\AttributeTok}[1]{\textcolor[rgb]{0.77,0.63,0.00}{#1}}
\newcommand{\RegionMarkerTok}[1]{#1}
\newcommand{\InformationTok}[1]{\textcolor[rgb]{0.56,0.35,0.01}{\textbf{\textit{#1}}}}
\newcommand{\WarningTok}[1]{\textcolor[rgb]{0.56,0.35,0.01}{\textbf{\textit{#1}}}}
\newcommand{\AlertTok}[1]{\textcolor[rgb]{0.94,0.16,0.16}{#1}}
\newcommand{\ErrorTok}[1]{\textcolor[rgb]{0.64,0.00,0.00}{\textbf{#1}}}
\newcommand{\NormalTok}[1]{#1}
\usepackage{longtable,booktabs}
\usepackage{graphicx,grffile}
\makeatletter
\def\maxwidth{\ifdim\Gin@nat@width>\linewidth\linewidth\else\Gin@nat@width\fi}
\def\maxheight{\ifdim\Gin@nat@height>\textheight\textheight\else\Gin@nat@height\fi}
\makeatother
% Scale images if necessary, so that they will not overflow the page
% margins by default, and it is still possible to overwrite the defaults
% using explicit options in \includegraphics[width, height, ...]{}
\setkeys{Gin}{width=\maxwidth,height=\maxheight,keepaspectratio}
\IfFileExists{parskip.sty}{%
\usepackage{parskip}
}{% else
\setlength{\parindent}{0pt}
\setlength{\parskip}{6pt plus 2pt minus 1pt}
}
\setlength{\emergencystretch}{3em}  % prevent overfull lines
\providecommand{\tightlist}{%
  \setlength{\itemsep}{0pt}\setlength{\parskip}{0pt}}
\setcounter{secnumdepth}{0}
% Redefines (sub)paragraphs to behave more like sections
\ifx\paragraph\undefined\else
\let\oldparagraph\paragraph
\renewcommand{\paragraph}[1]{\oldparagraph{#1}\mbox{}}
\fi
\ifx\subparagraph\undefined\else
\let\oldsubparagraph\subparagraph
\renewcommand{\subparagraph}[1]{\oldsubparagraph{#1}\mbox{}}
\fi

%%% Use protect on footnotes to avoid problems with footnotes in titles
\let\rmarkdownfootnote\footnote%
\def\footnote{\protect\rmarkdownfootnote}

%%% Change title format to be more compact
\usepackage{titling}

% Create subtitle command for use in maketitle
\newcommand{\subtitle}[1]{
  \posttitle{
    \begin{center}\large#1\end{center}
    }
}

\setlength{\droptitle}{-2em}

  \title{Practicals 5}
    \pretitle{\vspace{\droptitle}\centering\huge}
  \posttitle{\par}
  \subtitle{Factor Analysis}
  \author{David Raj Micheal}
    \preauthor{\centering\large\emph}
  \postauthor{\par}
    \date{}
    \predate{}\postdate{}
  

\begin{document}
\maketitle

\begin{Shaded}
\begin{Highlighting}[]
\CommentTok{#Load the data }
\KeywordTok{library}\NormalTok{(readxl)}
\NormalTok{data =}\StringTok{ }\KeywordTok{read_excel}\NormalTok{(}\StringTok{'nutritian.xlsx'}\NormalTok{)}
\KeywordTok{head}\NormalTok{(data)}
\end{Highlighting}
\end{Shaded}

\begin{longtable}[]{@{}rrrrrrr@{}}
\toprule
Protein\_g & Fat\_g & Carb\_g & Sugar\_g & VitB6\_mg & VitB12\_mcg &
VitE\_mg\tabularnewline
\midrule
\endhead
0.85 & 81.1 & 0.06 & 0.06 & 0.003 & 0.17 & 2.32\tabularnewline
0.85 & 81.1 & 0.06 & 0.06 & 0.003 & 0.13 & 2.32\tabularnewline
0.28 & 99.5 & 0.00 & 0.00 & 0.001 & 0.01 & 2.80\tabularnewline
21.40 & 28.7 & 2.34 & 0.50 & 0.166 & 1.22 & 0.25\tabularnewline
23.24 & 29.7 & 2.79 & 0.51 & 0.065 & 1.26 & 0.26\tabularnewline
20.75 & 27.7 & 0.45 & 0.45 & 0.235 & 1.65 & 0.24\tabularnewline
\bottomrule
\end{longtable}

\begin{Shaded}
\begin{Highlighting}[]
\CommentTok{#Load the psych package to perform factor analysis}

\KeywordTok{library}\NormalTok{(psych)}
\KeywordTok{head}\NormalTok{(data)}
\end{Highlighting}
\end{Shaded}

\begin{longtable}[]{@{}rrrrrrr@{}}
\toprule
Protein\_g & Fat\_g & Carb\_g & Sugar\_g & VitB6\_mg & VitB12\_mcg &
VitE\_mg\tabularnewline
\midrule
\endhead
0.85 & 81.1 & 0.06 & 0.06 & 0.003 & 0.17 & 2.32\tabularnewline
0.85 & 81.1 & 0.06 & 0.06 & 0.003 & 0.13 & 2.32\tabularnewline
0.28 & 99.5 & 0.00 & 0.00 & 0.001 & 0.01 & 2.80\tabularnewline
21.40 & 28.7 & 2.34 & 0.50 & 0.166 & 1.22 & 0.25\tabularnewline
23.24 & 29.7 & 2.79 & 0.51 & 0.065 & 1.26 & 0.26\tabularnewline
20.75 & 27.7 & 0.45 & 0.45 & 0.235 & 1.65 & 0.24\tabularnewline
\bottomrule
\end{longtable}

\begin{Shaded}
\begin{Highlighting}[]
\NormalTok{R =}\StringTok{ }\KeywordTok{cor}\NormalTok{(data)}
\NormalTok{fact.anal =}\StringTok{ }\KeywordTok{fa}\NormalTok{(R, }\DataTypeTok{nfactors =} \DecValTok{3}\NormalTok{, }\DataTypeTok{rotate=}\StringTok{"oblique"}\NormalTok{)}
\end{Highlighting}
\end{Shaded}

\begin{verbatim}
## Specified rotation not found, rotate='none' used
\end{verbatim}

\begin{verbatim}
## Warning in fac(r = r, nfactors = nfactors, n.obs = n.obs, rotate =
## rotate, : An ultra-Heywood case was detected. Examine the results carefully
\end{verbatim}

\begin{Shaded}
\begin{Highlighting}[]
\NormalTok{fact.anal}
\end{Highlighting}
\end{Shaded}

\begin{verbatim}
## Factor Analysis using method =  minres
## Call: fa(r = R, nfactors = 3, rotate = "oblique")
## Standardized loadings (pattern matrix) based upon correlation matrix
##              MR1   MR2   MR3   h2      u2 com
## Protein_g  -0.35  0.30  0.34 0.33  0.6725 2.9
## Fat_g       0.04  0.26 -0.24 0.13  0.8700 2.0
## Carb_g      0.87 -0.22  0.16 0.84  0.1603 1.2
## Sugar_g     0.64 -0.17  0.05 0.45  0.5546 1.2
## VitB6_mg    0.24  0.45  0.53 0.54  0.4636 2.4
## VitB12_mcg -0.09  0.26  0.34 0.19  0.8083 2.0
## VitE_mg     0.37  0.85 -0.37 1.00 -0.0016 1.8
## 
##                        MR1  MR2  MR3
## SS loadings           1.50 1.23 0.74
## Proportion Var        0.21 0.18 0.11
## Cumulative Var        0.21 0.39 0.50
## Proportion Explained  0.43 0.35 0.21
## Cumulative Proportion 0.43 0.79 1.00
## 
## Mean item complexity =  1.9
## Test of the hypothesis that 3 factors are sufficient.
## 
## The degrees of freedom for the null model are  21  and the objective function was  1.1
## The degrees of freedom for the model are 3  and the objective function was  0.03 
## 
## The root mean square of the residuals (RMSR) is  0.02 
## The df corrected root mean square of the residuals is  0.06 
## 
## Fit based upon off diagonal values = 0.99
## Measures of factor score adequacy             
##                                                    MR1  MR2  MR3
## Correlation of (regression) scores with factors   0.94 0.96 0.82
## Multiple R square of scores with factors          0.88 0.91 0.67
## Minimum correlation of possible factor scores     0.77 0.83 0.33
\end{verbatim}

\begin{Shaded}
\begin{Highlighting}[]
\NormalTok{fact.anal}\OperatorTok{$}\NormalTok{residual}
\end{Highlighting}
\end{Shaded}

\begin{longtable}[]{@{}lrrrrrrr@{}}
\toprule
& Protein\_g & Fat\_g & Carb\_g & Sugar\_g & VitB6\_mg & VitB12\_mcg &
VitE\_mg\tabularnewline
\midrule
\endhead
Protein\_g & 0.673 & 0.071 & 0.015 & -0.006 & -0.002 & 0.021 &
-0.025\tabularnewline
Fat\_g & 0.071 & 0.870 & 0.013 & 0.032 & -0.045 & -0.003 &
0.009\tabularnewline
Carb\_g & 0.015 & 0.013 & 0.160 & 0.007 & 0.002 & -0.015 &
-0.004\tabularnewline
Sugar\_g & -0.006 & 0.032 & 0.007 & 0.555 & -0.013 & 0.034 &
-0.010\tabularnewline
VitB6\_mg & -0.002 & -0.045 & 0.002 & -0.013 & 0.464 & -0.011 &
0.016\tabularnewline
VitB12\_mcg & 0.021 & -0.003 & -0.015 & 0.034 & -0.011 & 0.808 &
0.002\tabularnewline
VitE\_mg & -0.025 & 0.009 & -0.004 & -0.010 & 0.016 & 0.002 &
-0.002\tabularnewline
\bottomrule
\end{longtable}

\begin{Shaded}
\begin{Highlighting}[]
\CommentTok{#The following function does the factor analysis using maximum likelihood method}
\KeywordTok{factanal}\NormalTok{(data,}\DataTypeTok{factors =} \DecValTok{3}\NormalTok{,}\DataTypeTok{rotation =} \StringTok{"varimax"}\NormalTok{, }\DataTypeTok{scores =} \StringTok{"regression"}\NormalTok{)}
\end{Highlighting}
\end{Shaded}

\begin{verbatim}
## 
## Call:
## factanal(x = data, factors = 3, scores = "regression", rotation = "varimax")
## 
## Uniquenesses:
##  Protein_g      Fat_g     Carb_g    Sugar_g   VitB6_mg VitB12_mcg 
##      0.693      0.859      0.166      0.543      0.406      0.825 
##    VitE_mg 
##      0.005 
## 
## Loadings:
##            Factor1 Factor2 Factor3
## Protein_g  -0.341           0.429 
## Fat_g               0.367         
## Carb_g      0.913                 
## Sugar_g     0.673                 
## VitB6_mg    0.215   0.115   0.731 
## VitB12_mcg                  0.406 
## VitE_mg     0.117   0.969   0.205 
## 
##                Factor1 Factor2 Factor3
## SS loadings      1.475   1.096   0.933
## Proportion Var   0.211   0.157   0.133
## Cumulative Var   0.211   0.367   0.501
## 
## Test of the hypothesis that 3 factors are sufficient.
## The chi square statistic is 239 on 3 degrees of freedom.
## The p-value is 1.99e-51
\end{verbatim}


\end{document}
